%% LyX 2.3.6.1 created this file.  For more info, see http://www.lyx.org/.
%% Do not edit unless you really know what you are doing.
\documentclass[oneside,english]{amsart}
\usepackage[T1]{fontenc}
\usepackage[latin9]{inputenc}
\usepackage{geometry}
\geometry{verbose}
\usepackage{mathrsfs}
\usepackage{amsbsy}
\usepackage{amstext}
\usepackage{amsthm}

\makeatletter
%%%%%%%%%%%%%%%%%%%%%%%%%%%%%% Textclass specific LaTeX commands.
\numberwithin{equation}{section}
\numberwithin{figure}{section}

\makeatother

\usepackage{babel}
\begin{document}

\section{Single-patch Ebola model:}

\begin{align*}
S' & =\mu N-\beta_{I}SI-\beta_{W}SW-\mu S\\
I' & =\beta_{I}SI+\beta_{W}SW-\left(\gamma+\mu+\delta\right)I\\
R' & =\gamma I-\mu R\\
W' & =\xi_{\text{in}}I-\xi_{\text{out}}W
\end{align*}
where $N=S+I+R$.

Check that compartments remain positive, i.e. $\frac{dx_{i}}{dt}\ge0$
for any feasible set of initial conditions... (TBC)

Define $x=\left(I,W,S,R\right)^{t}$. There is a unique disease free
equilibrium given by $x_{0}=\left(N_{0},0,0,0\right)^{t}$, where
$N_{0}=N\left(0\right)$.

There are two infectious compartments: infected humans ($I$) and
the water reservoir ($W$), but new infections only occur in the infected
humans compartment.

Let $\mathscr{F}\left(x\right)$ the rate of new infections and $\mathscr{V}\left(x\right)$
the net rate of transfer for all other transitions.

\begin{align*}
\mathscr{F}\left(x\right) & =\begin{pmatrix}\beta_{I}SI+\beta_{W}SW\\
0\\
0\\
0
\end{pmatrix}\\
\mathscr{V}\left(x\right) & =\begin{pmatrix}\left(\gamma+\mu+\delta\right)I\\
\xi_{\text{out}}W-\xi_{\text{in}}I\\
\beta_{I}SI+\beta_{W}SW+\mu S-\mu N\\
\mu R-\gamma I
\end{pmatrix}
\end{align*}
The Jacobians of these 
\begin{align*}
D\mathscr{F}\left(x\right) & =\begin{bmatrix}\beta_{I}S & \beta_{W}S & \beta_{I}I+\beta_{W}W & 0\\
0 & 0 & 0 & 0\\
0 & 0 & 0 & 0\\
0 & 0 & 0 & 0
\end{bmatrix}\\
D\mathscr{V}\left(x\right) & =\begin{bmatrix}\gamma+\mu+\delta & 0 & 0 & 0\\
-\xi_{\text{in}} & \xi_{\text{out}} & 0 & 0\\
\beta_{I}S-\mu & \beta_{W}S & \beta_{I}I+\beta_{W}W & -\mu\\
-\gamma & 0 & 0 & \mu
\end{bmatrix}
\end{align*}
Taking the upper 2 by 2 quadrant of each matrix, to isolate the infected
compartments, and evaluating at the disease-free equilibrium, $x_{0}$,
gives us
\begin{align*}
F & =\begin{bmatrix}\beta_{I}N_{0} & \beta_{W}N_{0}\\
0 & 0
\end{bmatrix}\\
V & =\begin{bmatrix}\gamma+\mu+\delta & 0\\
-\xi_{\text{in}} & \xi_{\text{out}}
\end{bmatrix}
\end{align*}
The next generation, $K=FV^{-1}$, is given byHence
\begin{align*}
K=FV^{-1} & =\begin{bmatrix}\beta_{I}N_{0}\frac{1}{\gamma+\mu+\delta}+\beta_{W}N_{0}\frac{\xi_{\text{in}}}{\xi_{\text{out}}\left(\gamma+\mu+\delta\right)} & \frac{\beta_{W}N_{0}}{\xi}\\
0 & 0
\end{bmatrix}
\end{align*}


\subsection{Basic reproduction number}

The basic reproduction number, $\mathcal{R}_{0}$, defined as $\rho\left(K\right)$,
is therefore
\[
\mathcal{R}_{0}=\beta_{I}N_{0}\left(\frac{1}{\gamma+\mu+\delta}\right)+\beta_{W}N_{0}\left(\xi_{\text{in}}\frac{1}{\gamma+\mu+\delta}\right)\left(\frac{1}{\xi_{\text{out}}}\right)
\]
This can be decomposed as $\mathcal{R}_{0}=\mathcal{R}_{I}+\mathcal{R}_{W}$
where $\mathcal{R}_{I}$ represents the average number of new infections
induced by direct contact with an infectious individual (while alive)
and $\mathcal{R}_{W}$ represents the average number of new infections
induced by viral shedding of an infected individual into the water
source.

The type reproduction numbers, $\mathcal{R}_{I}$ and $\mathcal{R}_{W}$
can be interpreted as follows:
\begin{align*}
\mathcal{R}_{I} & =\beta_{I}N_{0}\left(\frac{1}{\gamma+\mu+\delta}\right)\\
 & =\frac{\text{\# infectious contacts}}{\text{person}\times\text{time}}\times\text{\# susceptibles}\times\text{infectious period}\\
\mathcal{R}_{W} & =\beta_{W}N_{0}\left(\xi_{\text{in}}\frac{1}{\gamma+\mu+\delta}\right)\left(\frac{1}{\xi_{\text{out}}}\right)\\
 & =\frac{\text{\# infectious contacts}}{\text{infectious units}\times\text{time}}\times\text{\# susceptibles}\times\left(\frac{\text{infectious viral units}}{\text{time}}\times\text{infectious period}\right)\times\text{virus persistence time}
\end{align*}

Note that here we have made$\xi_{\text{in}}$ and $\xi_{\text{out}}$
distinct to aid in the biological interpretation of the reproduction
number (otherwise the term $\frac{\xi_{\text{in}}}{\xi_{\text{out}}}=1$
would cancel out of the expression). 

\section{Two-patch Ebola model with migration and water movement}

\begin{align*}
S_{1}' & =\mu_{1}N_{1}-\beta_{I1}S_{1}I_{1}-\beta_{W1}S_{1}W_{1}-\mu_{1}S_{1}-m_{1}S_{1}+m_{2}S_{2}\\
I_{1}' & =\beta_{I1}S_{1}I_{1}+\beta_{W1}S_{1}W_{1}-(\gamma_{1}+\mu_{1}+\delta_{1})I_{1}-n_{1}I_{1}+n_{2}I_{2}\\
R_{1}' & =\gamma_{1}I_{1}-\mu_{1}R_{1}-m_{1}R_{1}+m_{2}R_{2}\\
W_{1}' & =\xi_{1,\text{in}}I_{1}-\xi_{1,\text{out}}W_{1}-\rho_{1}W_{1}+\rho_{2}W_{2}\\
S_{2}' & =\mu_{2}N_{2}-\beta_{I2}S_{2}I_{2}-\beta_{W2}S_{2}W_{2}-\mu_{2}S_{2}+m_{1}S_{1}-m_{2}S_{2}\\
I_{2}' & =\beta_{I2}S_{2}I_{2}+\beta_{W2}S_{2}W_{2}-(\gamma_{2}+\mu_{2}+\delta_{2})I_{2}+n_{1}I_{1}-n_{2}I_{2}\\
R_{2}' & =\gamma_{2}I_{2}-\mu_{2}R_{2}+m_{1}R_{1}-m_{2}R_{2}\\
W_{2}' & =\xi_{2,\text{in}}I_{2}-\xi_{2,\text{out}}W_{2}+\rho_{1}W_{1}-\rho_{2}W_{2}
\end{align*}
where $N_{i}=S_{i}+I_{i}+R_{i}$, $i=1,2$.

Check that compartments remain positive, i.e. $\frac{dx_{i}}{dt}\ge0$
for any feasible set of initial conditions... (TBC). Define $x=\left(I_{1},I_{2},W_{1},W_{2},S_{1},S_{2},R_{1},R_{2}\right)^{t}$.
There is a unique disease free equilibrium given by $x_{0}=\left(0,0,0,0,N_{1,0},N_{2,0},0,0\right)^{t}$,
where $N_{i,0}=N_{i}\left(0\right)$. There are two infectious compartments
in each patch: infected humans ($I_{1},I_{2}$) and the water reservoir
($W_{1},W_{2}$), but new infections only occur in the infected humans
compartments. Let $\mathscr{F}\left(x\right)$ the rate of new infections
and $\mathscr{V}\left(x\right)$ the net rate of transfer out of a
compartment (for all transitions except new infections).

\begin{align*}
\mathscr{F}\left(x\right) & =\begin{pmatrix}\beta_{I1}S_{1}I_{1}+\beta_{W1}S_{1}W_{1}\\
\beta_{I2}S_{2}I_{2}+\beta_{W2}S_{2}W_{2}\\
0\\
0\\
0\\
0\\
0\\
0
\end{pmatrix}\\
\mathscr{V}\left(x\right) & =\begin{pmatrix}(\gamma_{1}+\mu_{1}+\delta_{1})I_{1}+n_{1}I_{1}-n_{2}I_{2}\\
-n_{1}I_{1}+(\gamma_{2}+\mu_{2}+\delta_{2})I_{2}+n_{2}I_{2}\\
-\xi_{1,\text{in}}I_{1}+\xi_{1,\text{out}}W_{1}+\rho_{1}W_{1}-\rho_{2}W_{2}\\
-\rho_{1}W_{1}-\xi_{2,\text{in}}I_{2}+\xi_{2,\text{out}}W_{2}+\rho_{2}W_{2}\\
\beta_{I1}S_{1}I_{1}+\beta_{W1}S_{1}W_{1}-\mu_{1}N_{1}+\mu_{1}S_{1}+m_{1}S_{1}-m_{2}S_{2}\\
-\gamma_{1}I_{1}+\mu_{1}R_{1}+m_{1}R_{1}-m_{2}R_{2}\\
\beta_{I2}S_{2}I_{2}+\beta_{W2}S_{2}W_{2}-\mu_{2}N_{2}+\mu_{2}S_{2}-m_{1}S_{1}+m_{2}S_{2}\\
-m_{1}R_{1}-\gamma_{2}I_{2}+\mu_{2}R_{2}+m_{2}R_{2}
\end{pmatrix}
\end{align*}
The Jacobians of these operators, restricted to the rows and columns
representing infected compartments, and evaluated at the DFE, $x_{0}$,
are given by the matrices
\begin{align*}
F & =\begin{bmatrix}\beta_{I1}N_{1,0} & 0 & \beta_{W1}N_{1,0} & 0\\
0 & \beta_{I2}N_{2,0} & 0 & \beta_{W2}N_{2,0}\\
0 & 0 & 0 & 0\\
0 & 0 & 0 & 0
\end{bmatrix}\\
V & =\begin{bmatrix}\gamma_{1}+\mu_{1}+\delta_{1}+n_{1} & -n_{2} & 0 & 0\\
-n_{1} & \gamma_{2}+\mu_{2}+\delta_{2}+n_{2} & 0 & 0\\
-\xi_{1,\text{in}} & 0 & \xi_{1,\text{out}}+\rho_{1} & -\rho_{2}\\
0 & -\xi_{2,\text{in}} & -\rho_{1} & \xi_{2,\text{out}}+\rho_{2}
\end{bmatrix}
\end{align*}
To simplify our calculations, we can re-write these in block form:
\begin{align*}
F & =\begin{bmatrix}\boldsymbol{F_{I}} & \boldsymbol{F_{W}}\\
\boldsymbol{0_{2}} & \boldsymbol{0_{2}}
\end{bmatrix},V=\begin{bmatrix}\boldsymbol{V_{I}} & \boldsymbol{0_{2}}\\
\boldsymbol{D} & \boldsymbol{V_{W}}
\end{bmatrix}
\end{align*}
 where 
\begin{align*}
\boldsymbol{F_{I}} & =\begin{bmatrix}\beta_{I1}N_{1,0} & 0\\
0 & \beta_{I2}N_{2,0}
\end{bmatrix},\\
\boldsymbol{F_{W}} & =\begin{bmatrix}\beta_{W1}N_{1,0} & 0\\
0 & \beta_{W2}N_{2,0}
\end{bmatrix},\\
\boldsymbol{V_{I}} & =\begin{bmatrix}\gamma_{1}+\mu_{1}+\delta_{1}+n_{1} & -n_{2}\\
-n_{1} & \gamma_{2}+\mu_{2}+\delta_{2}+n_{2}
\end{bmatrix},\\
\boldsymbol{V_{W}} & =\begin{bmatrix}\xi_{1,\text{out}}+\rho_{1} & -\rho_{2}\\
-\rho_{1} & \xi_{2,\text{out}}+\rho_{2}
\end{bmatrix},\\
\boldsymbol{D} & =\begin{bmatrix}-\xi_{1,\text{in}} & 0\\
0 & -\xi_{2,\text{in}}
\end{bmatrix}.
\end{align*}
Note that each of $\boldsymbol{V_{I}}$, $\boldsymbol{V_{W}}$, and
$\boldsymbol{D}$ are invertible with biologically feasible parameters
(i.e. not all are zero). The inverse of $V$ can also be written in
block form as:
\[
V^{-1}=\begin{bmatrix}\boldsymbol{V_{I}}^{-1} & \boldsymbol{0_{2}}\\
-\boldsymbol{V_{W}}^{-1}\boldsymbol{D}\boldsymbol{V_{I}}^{-1} & \boldsymbol{V_{W}}^{-1}
\end{bmatrix}
\]
and the next generation matrix, $K=FV^{-1}$, can be determined in
block-matrix form:
\begin{align*}
K=FV^{-1} & =\begin{bmatrix}\boldsymbol{F_{I}} & \boldsymbol{F_{W}}\\
\boldsymbol{0_{2}} & \boldsymbol{0_{2}}
\end{bmatrix}\begin{bmatrix}\boldsymbol{V_{I}}^{-1} & \boldsymbol{0_{2}}\\
-\boldsymbol{V_{W}}^{-1}\boldsymbol{D}\boldsymbol{V_{I}}^{-1} & \boldsymbol{V_{W}}^{-1}
\end{bmatrix}\\
 & =\begin{bmatrix}\boldsymbol{F_{I}}\boldsymbol{V_{I}}^{-1}-\boldsymbol{F_{W}}\boldsymbol{V_{W}}^{-1}\boldsymbol{D}\boldsymbol{V_{I}}^{-1} & \boldsymbol{F_{W}}\boldsymbol{V_{W}}^{-1}\\
\boldsymbol{0_{2}} & \boldsymbol{0_{2}}
\end{bmatrix}
\end{align*}
We see immediately that $K$ will have at least two zero eigenvalues.
The remaining eigenvalues are the eigenvalues of $K_{I}=\boldsymbol{F_{I}}\boldsymbol{V_{I}}^{-1}-\boldsymbol{F_{W}}\boldsymbol{V_{W}}^{-1}\boldsymbol{D}\boldsymbol{V_{I}}^{-1}$.
We proceed in computing the eigenvalues of $K_{I}$ by first determining
$\boldsymbol{V_{I}}^{-1}$ and $\boldsymbol{V_{W}}^{-1}$.

We will work through the determination of $\boldsymbol{V_{I}}^{-1}$
below in a deliberate manner in order to facilitate the biological
interpretation of the next generation matrix and the basic reproduction
number. We start by introducing some new notation. Let $\lambda_{i}=\frac{1}{\gamma_{i}+\mu_{i}+\delta_{i}+n_{i}}$
and $\sigma_{i}=\frac{n_{i}}{\gamma_{i}+\mu_{i}+\delta_{i}+n_{i}}=n_{i}\lambda_{i}$,
$i=1,2$. The quantity $\lambda_{i}$ represents the average amount
of time an infectious individual in patch $i$ will remain in patch
$i$ before recovering, dying (due to disease or natural mortality),
or migrating. The quantity $\sigma_{i}$ represents the probability
that an infectious individual in patch $i$ will migrate to another
patch before recovering or dying. With these definitions, we can write
$\boldsymbol{V_{I}}^{-1}$ much more compactly:
\begin{align*}
\boldsymbol{V_{I}}^{-1} & =\left(\frac{1}{1-\sigma_{1}\sigma_{2}}\right)\begin{bmatrix}\lambda_{1} & 0\\
0 & \lambda_{2}
\end{bmatrix}\begin{bmatrix}1 & \sigma_{2}\\
\sigma_{1} & 1
\end{bmatrix}
\end{align*}
Note that since $\sigma_{1}$ and $\sigma_{2}$ are probabilities,
the quantity $\frac{1}{1-\sigma_{1}\sigma_{2}}$ is equivalent to
the series $\sum_{n=1}^{\infty}\left(\sigma_{1}\sigma_{2}\right)^{n}$,
which gives the average number of migrations an infectious individual
makes before either recovering or dying. Note that this quantity is
the same whether an individual starts in patch $1$ or in patch $2$,
thus we can let $\nu=\frac{1}{1-\sigma_{1}\sigma_{2}}$ and drop the
subscript.

Notice that $\boldsymbol{V_{W}}$ is structured very similarly to
$\boldsymbol{V_{I}}$. Thus we can immediately determine the inverse
of $\boldsymbol{V_{W}}$ after defining some quantities similar to
those in $\boldsymbol{V_{I}}^{-1}$. Let $\eta_{i}=\frac{1}{\xi_{i,\text{out}}+\rho_{i}}$,
$\omega_{i}=\frac{\rho_{i}}{\xi_{i,\text{out}}+\rho_{i}}$, and $\tau=\frac{1}{1-\omega_{1}\omega_{2}}$.
$\eta_{i}$, $\omega_{i}$, and $\tau$ have equivalent interpretations
to $\lambda_{i}$, $\sigma_{i}$, and $\nu$ but for the transport
of virus shed in the water, as opposed to human migration. With these
defined, we can write down the inverse of $\boldsymbol{V_{W}}$:
\[
\boldsymbol{V_{W}}^{-1}=\tau\begin{bmatrix}\eta_{1} & 0\\
0 & \eta_{2}
\end{bmatrix}\begin{bmatrix}1 & \omega_{2}\\
\omega_{1} & 1
\end{bmatrix}
\]

Now to compute $K_{I}$, note that $\boldsymbol{F_{I}}\boldsymbol{V_{I}}^{-1}-\boldsymbol{F_{W}}\boldsymbol{V_{W}}^{-1}\boldsymbol{D}\boldsymbol{V_{I}}^{-1}=\left(\boldsymbol{F_{I}}-\boldsymbol{F_{W}}\boldsymbol{V_{W}}^{-1}\boldsymbol{D}\right)\boldsymbol{V_{I}}^{-1}$.
Computing the bracketed portion first:
\begin{align*}
\boldsymbol{F_{I}}-\boldsymbol{F_{W}}\boldsymbol{V_{W}}^{-1}\boldsymbol{D} & =\begin{bmatrix}\beta_{I1}N_{1,0} & 0\\
0 & \beta_{I2}N_{2,0}
\end{bmatrix}-\begin{bmatrix}\beta_{W1}N_{1,0} & 0\\
0 & \beta_{W2}N_{2,0}
\end{bmatrix}\tau\begin{bmatrix}\eta_{1} & 0\\
0 & \eta_{2}
\end{bmatrix}\begin{bmatrix}1 & \omega_{2}\\
\omega_{1} & 1
\end{bmatrix}\begin{bmatrix}-\xi_{1,\text{in}} & 0\\
0 & -\xi_{2,\text{in}}
\end{bmatrix}\\
 & =\begin{bmatrix}\beta_{I1}N_{1,0}+\tau\beta_{W1}N_{1,0}\eta_{1}\xi_{1,\text{in}} & \tau\beta_{W1}N_{1,0}\eta_{1}\omega_{2}\xi_{2,\text{in}}\\
\tau\beta_{W2}N_{2,0}\eta_{2}\omega_{1}\xi_{1,\text{in}} & \beta_{I2}N_{2,0}+\tau\beta_{W2}N_{2,0}\eta_{2}\xi_{2,\text{in}}
\end{bmatrix}
\end{align*}
Finally,
\begin{align*}
K_{I} & =\begin{bmatrix}k_{11} & k_{12}\\
k_{21} & k_{22}
\end{bmatrix}
\end{align*}
where 
\begin{align*}
k_{11} & =\beta_{I1}N_{1,0}\lambda_{1}\nu+\beta_{W1}N_{1,0}\xi_{1,\text{in}}\lambda_{1}\left(\tau\eta_{1}\right)\nu+\beta_{W1}N_{1,0}\xi_{2,\text{in}}\lambda_{2}\left(\tau\eta_{1}\omega_{2}\right)\left(\nu\sigma_{1}\right),\\
k_{12} & =\beta_{I1}N_{1,0}\lambda_{1}\nu\sigma_{2}+\beta_{W1}N_{1,0}\xi_{1,\text{in}}\lambda_{1}\left(\tau\eta_{1}\right)\left(\nu\sigma_{2}\right)+\beta_{W1}N_{1,0}\xi_{2,\text{in}}\lambda_{2}\left(\tau\eta_{1}\omega_{2}\right)\nu,\\
k_{21} & =\beta_{I2}N_{2,0}\lambda_{2}\nu\sigma_{1}+\beta_{W2}N_{2,0}\xi_{2,\text{in}}\lambda_{2}\left(\tau\eta_{2}\right)\left(\nu\sigma_{1}\right)+\beta_{W2}N_{2,0}\xi_{1,\text{in}}\lambda_{1}\left(\tau\eta_{2}\omega_{1}\right)\nu,\\
k_{22} & =\beta_{I2}N_{2,0}\lambda_{2}\nu+\beta_{W2}N_{2,0}\xi_{2,\text{in}}\lambda_{2}\left(\tau\eta_{2}\right)\nu+\beta_{W2}N_{2,0}\xi_{1,\text{in}}\lambda_{1}\left(\tau\eta_{2}\omega_{1}\right)\left(\nu\sigma_{2}\right).
\end{align*}
We will argue that $k_{ij}$ represents the average number of new
infections induced in patch $i$ by an infectious individual who started
in patch $j$. Define
\begin{align*}
\mathcal{R}_{Ii} & =\beta_{Ii}N_{i,0}\lambda_{i},\\
\mathcal{R}_{Wi} & =\beta_{Wi}N_{i,0}\xi_{i,\text{in}}\lambda_{i}\eta_{i}.
\end{align*}
These are the within-patch type reproduction numbers when there is
no human migration or water transportation. It is also helpful to
recall the interpretations of the several quantities introduced above:
\begin{align*}
\nu & =\text{average number of "there and back" migrations before recovering or dying}\\
\lambda_{i} & =\text{average infectious period of an infected individual remaining in patch \ensuremath{i}}\\
\sigma_{i} & =\text{probability of migrating out of patch \ensuremath{i} before recovering or dying }\\
\tau & =\text{average number of "there and back" transportations of shed virus in water before decaying}\\
\eta_{i} & =\text{average lifetime of shed virus in the water of patch \ensuremath{i}}\\
\omega_{i} & =\text{probability of shed virus being transported out of patch \ensuremath{i} before decaying}
\end{align*}
Then we can re-write the entries of $K$ as:
\begin{align*}
k_{11} & =\nu\mathcal{R}_{I1}+\nu\tau\mathcal{R}_{W1}+\left(\sigma_{1}\nu\right)\left(\omega_{2}\tau\right)\frac{\xi_{2,\text{in}}\lambda_{2}}{\xi_{1,\text{in}}\lambda_{1}}\mathcal{R}_{W1}\\
k_{12} & =\sigma_{2}\nu\mathcal{R}_{I1}+\left(\sigma_{2}\nu\right)\tau\mathcal{R}_{W1}+\nu\left(\omega_{2}\tau\right)\frac{\xi_{2,\text{in}}\lambda_{2}}{\xi_{1,\text{in}}\lambda_{1}}\mathcal{R}_{W1}\\
k_{22} & =\sigma_{1}\nu\mathcal{R}_{I2}+\left(\sigma_{1}\nu\right)\tau\mathcal{R}_{W2}+\nu\left(\omega_{1}\tau\right)\frac{\xi_{1,\text{in}}\lambda_{1}}{\xi_{2,\text{in}}\lambda_{2}}\mathcal{R}_{W2}\\
k_{22} & =\nu\mathcal{R}_{I2}+\nu\tau\mathcal{R}_{W2}+\left(\nu\sigma_{2}\right)\left(\tau\omega_{1}\right)\frac{\xi_{1,\text{in}}\lambda_{1}}{\xi_{2,\text{in}}\lambda_{2}}\mathcal{R}_{W2}
\end{align*}

The quantities $k_{ij}$ gives the average number of new infections
in patch $i$ induced by an infected individual initially in patch
$j$. Each $k_{ij}$ is composed of three terms which correspond to
the three ways this individual may have contributed to infections
in patch $i$: direct transmission after migrating to or starting
in patch $i$, environmental transmission after migrating to patch
$i$ and shedding virus there, and finally environmental transmission
from shedding virus in patch $j$ which is then transported by the
water to patch $i$.

So, for example, looking at $k_{12}$, the term $\sigma_{2}\nu\mathcal{R}_{I1}$
represents the average number of new infections induced in patch $1$
by an infectious individual who started in patch $2$ then migrated
to patch $1$ (with probability $\sigma_{2}$), including all ``there
and back'' migrations ($\nu$). The next term, $\left(\sigma_{2}\nu\right)\tau\mathcal{R}_{W1}$
gives the number of new infections induced in patch $1$ through the
water caused by an infected individual who started in patch $2$ and
migrated before recovery or death ($\sigma_{2}$), then while in there
in patch $1$, shed virus into the water, which leads to on average,
$\mathcal{R}_{W1}$ new infections. Finally, $\nu\left(\omega_{2}\tau\right)\frac{\xi_{2,\text{in}}\lambda_{2}}{\xi_{1,\text{in}}\lambda_{1}}\mathcal{R}_{W1}$
gives the number of infections induced by an infectious individual
in patch $2$ whose shed virus is transported to patch $1$ in the
water, with the term $\frac{\xi_{2,\text{in}}\lambda_{2}}{\xi_{1,\text{in}}\lambda_{1}}$
accounting for differences in the flow of water and infectious periods
in each patch.

Note that when migration and water transport are turned off in one
direction, then the corresponding type reproduction number is zero.
For example, if there is no movement of individuals or water say movement
from patch $1$ to patch $2$ (i.e. $\sigma_{1}=\omega_{1}=0$), then
$k_{12}=0$. Thus we rename the entries of $K_{i}$ to $\mathcal{R}_{ij}=k_{ij}$.
With this notation, we can easily write down the basic reproduction
number of the full system in terms of these patch-reproduction numbers:
\begin{align*}
\mathcal{R}_{0} & =\frac{1}{2}\left(\mathcal{R}_{11}+\mathcal{R}_{22}\right)+\frac{1}{2}\sqrt{\left(\mathcal{R}_{11}-\mathcal{R}_{22}\right)^{2}+4\mathcal{R}_{12}\mathcal{R}_{21}}
\end{align*}


\subsection{Some initial observations about the basic reproduction number}
\begin{itemize}
\item If there is no movement whatsoever of people or water ($\sigma_{i}=\omega_{i}=0$),
then $\mathcal{R}_{0}=\max\left\{ \mathcal{R}_{11},\mathcal{R}_{22}\right\} $.
\item In general, $\max\left\{ \mathcal{R}_{11},\mathcal{R}_{22}\right\} \le\mathcal{R}_{0}\le\max\left\{ \mathcal{R}_{11},\mathcal{R}_{22}\right\} +\sqrt{\mathcal{R}_{12}\mathcal{R}_{21}}$
\end{itemize}

\subsection{Water flowing in a single direction}

This scenario has water only flowing out of patch $1$, into patch
$2$, then out of patch $2$ but \textbf{not }back in to patch $1$,
we would need to rename our variables. This changes the structure
of the matrix $V$. Unfortunately we cannot check this easily (say,
by setting some parameters to zero) because the original model is
structurally different from the one represented in this scenario i.e.
we cannot set ``one'' of the $\rho_{2}$s to zero with the other
positive.

But, just as an example, suppose $\rho_{2}=0$. Then $\omega_{2}=0$
and $\tau=1$ and we get
\begin{align*}
\mathcal{R}_{11} & =\nu\mathcal{R}_{I1}+\eta_{1}\nu\mathcal{R}_{W1}\\
\mathcal{R}_{21} & =\nu\sigma_{2}\mathcal{R}_{I1}+\eta_{1}\nu\sigma_{2}\mathcal{R}_{W1}\\
\mathcal{R}_{12} & =\nu\sigma_{1}\mathcal{R}_{I2}+\eta_{2}\nu\left[\left(\sigma_{1}\right)+\frac{\xi_{1,\text{in}}\lambda_{1}}{\xi_{2,\text{in}}\lambda_{2}}\left(\omega_{1}\right)\right]\mathcal{R}_{W2}\\
\mathcal{R}_{22} & =\nu\mathcal{R}_{I2}+\eta_{2}\nu\left[1+\frac{\xi_{1,\text{in}}\lambda_{1}}{\xi_{2,\text{in}}\lambda_{2}}\left(\omega_{1}\right)\left(\sigma_{2}\right)\right]\mathcal{R}_{W2}
\end{align*}

\end{document}
